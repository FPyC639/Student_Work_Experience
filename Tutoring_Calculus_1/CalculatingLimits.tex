\documentclass{article}
\usepackage{amsmath}
\usepackage{cancel}

%\title{Essential Calculus Early Transcendentals Second Edition}
\title{Calculating Limits}

\author{Jose M Serra Jr}

\begin{document}
	
	\maketitle
	\begin{abstract}
		\indent\indent In this section I will be going over how to calculate limits on a particular set of problems. This solutions are made solely by myself and used for demostration.
	\end{abstract}
	\newpage
	Number 11:
	\begin{equation*}
		\lim\limits_{x\to 5}{\frac{x^2-6x+5}{x-5}}
	\end{equation*}
	\begin{eqnarray*}
		\lim\limits_{x\to 5}{\frac{x^2-6x+5}{x-5}} &= L \\
		\lim\limits_{x\to 5}{\frac{(x-5)(x-1)}{x-5}} &= L \\
		\lim\limits_{x\to 5}{\frac{\cancel{(x-5)}(x-1)}{\cancel{(x-5)}}} &= L \\
		\lim\limits_{x\to 5}{(x-1)} &= L \\
		(x-1)\Bigg\vert_{x=5} = (5-1) = 4 &= L
	\end{eqnarray*}
	\textbf{Solution:}
	\begin{equation*}
		\lim\limits_{x\to 5}{\frac{x^2-6x+5}{x-5}}=4
	\end{equation*}
	Number 12:
	\begin{equation*}
		\lim\limits_{x\to 4}{\frac{x^2-4x}{x^2-3x-4}}
	\end{equation*}
	\begin{eqnarray*}
		\lim\limits_{x\to 4}{\frac{x^2-4x}{x^2-3x-4}} &= L \\
		\lim\limits_{x\to 4}{\frac{x(x-4)}{(x+1)(x-4)}} &= L \\
		\lim\limits_{x\to 4}{\frac{x\cancel{(x-4)}}{(x+1)\cancel{(x-4)}}} &= L \\
		\lim\limits_{x\to 4}{\frac{x}{x+1}} &= L \\
		\frac{x}{x+1}\Bigg\vert_{x=4} = \frac{4}{4+1}= \frac{4}5 &= L
	\end{eqnarray*}
	\textbf{Solution:}
	\begin{equation*}
		\lim\limits_{x\to 4}{\frac{x^2-4x}{x^2-3x-4}} = \frac45
	\end{equation*}
	Number 13:
	\begin{equation*}
		\lim\limits_{x\to 5}{x^2-5x+6}{x-5}
	\end{equation*}
	\begin{eqnarray*}
		\lim\limits_{x\to 5}{\frac{x^2-5x+6}{x-5}} &= L \\
		\frac{x^2-5x+6}{x-5}\Bigg\vert_{x=5} = \frac{(5)^2-5(5)+6}{5-5} = \frac{6}{0} &= L
	\end{eqnarray*}
	\textbf{Solution:}
	Since it is not possible to divide zero the limit does not exist.
	\newline
	\newline
	Number 14:
	\begin{equation*}
		\lim\limits_{x\to -1}{\frac{x^2-4x}{x^2-3x-4}}
	\end{equation*}
	\begin{eqnarray*}
		\lim\limits_{x\to -1}{\frac{x^2-4x}{x^2-3x-4}} &= L \\ \frac{x^2-4x}{x^2-3x-4}\Bigg\vert_{x=-1} = \frac{(-1)^2-4(-1)}{(-1)^2-3(-1)-4}=\frac50 &= L
	\end{eqnarray*}
	\textbf{Solution:}
	Since it is not possible to divide by zero therefore the limit does not exist.
	\newline
	\newline
	Number 15:
	\begin{equation*}
		\lim\limits_{t\to -3}{\frac{t^2-9}{2t^2+7t+3}}
	\end{equation*}
	\begin{eqnarray*}
		\lim\limits_{t\to -3}{\frac{t^2-9}{2t^2+7t+3}}&=L \\ \lim\limits_{t\to -3}{\frac{(t+3)(t-3)}{(2t+1)(x+3)}} &= L \\
		\lim\limits_{t\to -3}{\frac{\cancel{(t+3)}(t-3)}{(2t+1)\cancel{(x+3)}}} &= L \\
		\frac{t-3}{2t+1}\Bigg\vert_{t=-3} = \frac{-3-3}{2(-3)+1}=\frac{6}{5} &= L
	\end{eqnarray*}
	\textbf{Solution:}
	\begin{equation*}
		\lim\limits_{t\to -3}{\frac{t^2-9}{2t^2+7t+3}} = \frac65
	\end{equation*}
	Number 16:
	\begin{equation*}
		\lim\limits_{x\to -1}{\frac{2x^2+3x+1}{x^2-2x-3}}
	\end{equation*}
	\begin{eqnarray*}
		\lim\limits_{x\to -1}{\frac{2x^2+3x+1}{x^2-2x-3}} &= L \\
		\lim\limits_{x\to -1}{\frac{(2x+1)(x+1)}{(x+1)(x-3)}} &= L \\
		\lim\limits_{x\to -1}{\frac{(2x+1)\cancel{(x+1)}}{\cancel{(x+1)}(x-3)}} &= L \\
		\frac{2x+1}{x-3}\Bigg\vert_{x=-1} = \frac{(2(-1)+1)}{(-1)-3}= \frac14 &=L
	\end{eqnarray*}
	\textbf{Solution:}
	\begin{equation*}
		\lim\limits_{x\to -1}{\frac{2x^2+3x+1}{x^2-2x-3}}=\frac14
	\end{equation*}
	Number 17
	\begin{equation*}
		\lim\limits{h\to 0}\frac{(-5+h)^2-25}{h}
	\end{equation*}
	\begin{eqnarray*}
		\lim\limits_{h\to 0}\frac{(-5+h)^2-25}{h} &= L \\
		\lim\limits_{h\to 0}\frac{25-10h+h^2-25}{h} &= L\\
		\lim\limits_{h\to 0}\frac{h(-10+h)}{h} &= L \\
		\lim\limits_{h\to 0}\frac{\cancel{h}(-10+h)}{\cancel{h}} &= L \\
		(-10+h)\Bigg\vert_{h=0} = (-10+0) = -10 &= L
	\end{eqnarray*}
	\textbf{Solution:}
	\begin{equation*}
		\lim\limits{h\to 0}\frac{(-5+h)^2-25}{h} = -10
	\end{equation*}
	Number 18
	\begin{equation*}
		\lim\limits_{h\to 0}\frac{(2+h)^3-8}{h}
	\end{equation*}
	\begin{eqnarray*}
		\lim\limits_{h\to 0}\frac{(2+h)^3-8}{h} &= L \\
		\lim\limits_{h\to 0}\frac{8+4h+2h^2+h^3-8}{h} &= L\\
		\lim\limits_{h\to 0}\frac{h(4+2h+h^2)}{h} &= L\\
		\lim\limits_{h\to 0}\frac{\cancel{h}(4+2h+h^2)}{\cancel{h}} &= L \\
		(4+2h+h^2)\Bigg\vert_{h=0} = 4+0+0= 4 &= L
	\end{eqnarray*}
	\textbf{Solution:}
	\begin{equation*}
		\lim\limits_{h\to 0}\frac{(2+h)^3-8}{h} = 4
	\end{equation*}
	Number 19
	\begin{equation*}
		\lim\limits{x\to-2}{\frac{x+2}{x^3+8}}
	\end{equation*}
	\begin{eqnarray*}
		\lim\limits_{x\to-2}{\frac{x+2}{(x+2)(x^2-2x+4)}}&=L\\
			\lim\limits_{x\to-2}{\frac{\cancel{x+2}}{\cancel{(x+2)}(x^2-2x+4)}} &= L \\
			\lim\limits_{x\to-2}{\frac{1}{x^2-2x+4}} &=L \\
			\frac{1}{x^2-2x+4}\Bigg\vert_{x=-2}=\frac{1}{4+4+4}=\frac{1}{12}&=L
	\end{eqnarray*}
	\textbf{Solution:}
	\begin{equation*}
		\lim\limits_{x\to-2}{\frac{x+2}{x^3+8}} = \frac{1}{12}
	\end{equation*}
\end{document}